\documentclass[a4paper,12pt]{article}
\usepackage{graphicx}
\usepackage[left=30mm, right=30mm, top=30mm, bottom=35mm]{geometry}
\usepackage{amsmath}
\usepackage{siunitx}
\usepackage{fancyhdr}
\usepackage{url}
\pagestyle{fancy}
%-------------------------------------------------------------------------------
\lhead{\textbf{Fall 2019}}
\rhead{\textbf{CE311K Intro to Computer Methods}}
\cfoot{\thepage}
%-------------------------------------------------------------------------------

\begin{document}
\begin{centering}
	\textbf{
		Assignment 05: Vectors and Matrices\\
		Assigned: 19th November 2019\\
		Due: 6th December 2019 at 5 PM\\
	}
\end{centering}


Note: Please upload your solution as an ipynb file to the Canvas page.

\vspace{1em}
 
 The purpose of this assignment is to develop your skills in using Numpy arrays and doing vector and matrix operations.
 
\begin{enumerate}
	\item Let $A$ be a 4 x 4 matrix and B a 2 x 1 matrix. The symbol $\cdot$ represents a dot product. Using the shape of the matrix as a guide, indicate if the following expressions are valid or not.
	\begin{enumerate}
		\item $ A + B$
		\item $ A \cdot B$
		\item $ A \cdot A$
		\item $ B \cdot B$
		\item $ B^T \cdot B \cdot A$
	\end{enumerate}

	\item Using the following matrices and vectors. Compute the following.
	
	\begin{align*}
	A & =  \begin{bmatrix}
		1 & 4 & -2 \\
		4 & 8 & 6 \\
		-2 & 6 & 12 
    	\end{bmatrix} \\
	%
	B & =  \begin{bmatrix}
	6 & 2 & -2 \\
	4 & 8 & 3 \\
	-1 & 6 & 9 
	\end{bmatrix} \\
	c & = \left[6, -4, 3\right]\\
	d & = \left[3, -1, 5\right]
	\end{align*}
	\begin{enumerate}
		\item $ A - A^T$
		\item $A \cdot B$
		\item $8A - 4B$
		\item $B^T \cdot B$
		\item $c^T \cdot B$
		\item $A\cdot B \cdot c \cdot d^T$
		\item $c \times d$
		\item Inverse $A^{-1}$ and check if $A * A^{-1} = I$
	\end{enumerate}
	\item Find a a unit vector (i.e., vector of magnitude equal to 1) that is perpendicular to both $c$ and $d$. Hint: \textit{use the cross product definition}.
	\item Using array slicing on matrix A
	\begin{align*}
	A = \begin{bmatrix}
4.0 & 7.0 & -2.43 & 67.1 \\
-4.0 & 64.0 & 54.7 & -3.33 \\
2.43 & 23.2 & 3.64 & 4.11 \\
1.2 & 2.5 & -113.2 & 323.22
	\end{bmatrix}
	\end{align*}
	\begin{enumerate}
		\item Extract the third column as a 1D array
		\item Extract the first two rows as a 2D sub-array
		\item Extract the bottom-right $2 \times 2$ block as a 2D sub-array
		\item Sum the last column
	\end{enumerate}
\end{enumerate}

\end{document}


\documentclass[]{article}

\usepackage{url}
\begin{document}
\begin{center}
\Large Finals Study guide
\end{center}

Finals for CE 311K is a three-hour closed-book exam held on \textbf{17th December 2019} in \textbf{Graduate School of Business Building 2.124} \url{https://utdirect.utexas.edu/apps/campus/buildings/nlogon/facilities/UTM/GSB/} between \textbf{9AM and 12PM}. You may bring three sheets of 8.5 x 11 inch of your own handwritten notes to the examination. You may use your calculator. You may not access the internet during the exam. The exam questions will be determined such that they satisfy a subset of the objectives listed here.

Finals will cover:
\begin{itemize}
	\item Lecture Introduction, Control flow, errors, functions, data structures (only lists), Taylor series and Newton Iterations, vectors, matrices, solution of linear system of equations. \textit{Tuples, Dictionary, Exceptions and Useful libraries are not part of the exam}
	\item Relevant assignments and labs
\end{itemize}

To perform successfully on the finals, you should be able to:

\begin{enumerate}
	\item Determine if a given Matrix-Vector operation is valid or not (for e.g., $A\cdot b$, where $A$ is a 3x3 matrix and $b$ is a 3x1 vector).
	\item Use array slicing in vectors and matrices to get selected sub-set of elements, rows or columns.
	\item Perform Gauss Elimination by hand for a given set of 3 linear equations.
	\item Perform a maximum of three iterations of Gauss-Seidel for a given system of 3 linear equations by hand.
	\item Evaluate if a given matrix $A$ can be solved using Gauss-Seidel iterative approach.
	\item Construct the system of linear equations of a truss in a matrix format. Linear equations will be provided and there is no need to solve the system of equations.
	\item Exceptions are not part of the exam.
	\item Useful libs are not part of the exam.
\end{enumerate}

In addition, Finals will also cover the topics from Exams I and II:

\begin{enumerate}
	\item Determine data types (\verb|bool|, \verb|int|, \verb|float|, \verb|string|) of different operations (for e.g, $*, +, -, /, \%, //$)
	\item Write simple programs using variables and evaluate the value bound to a variable(s) at different step(s) in the code.
	\item Understand and evaluate the order of precedence in a given statement(s).
	\item Identify syntax, semantic and static-semantic errors in the code. It is not required to identify the exact type of error in the code, but the location of an error(s) in the code.
	\item Identify and fix incorrect code either using the output error message, verification result or logic.
	\item Understand and develop logic / algorithms / code that involve iterations (single and nested \verb|for|, \verb|for - else| and \verb|range|) and control flow (\verb|if|, \verb|elif|, \verb|else|, \verb|break| and \verb|continue|).
	\item Understand the output of a given code or expression (for e.g., \verb|range()|)
	\item Explain in your own word what are relative and absolute errors.
	\item Evaluate relative and absolute error(s) for a given program. 
	\item Evaluate a suitable termination criteria (\verb|break|) based on the relative or absolute error.	
	\item Determine data types and outputs during casting operations. Please note only Python standard data types (\verb|str|, \verb|int|, \verb|float|, \verb|bool|) will be covered. Numpy data types are not included (for e.g., \verb|np.float16| and others)
	\item Define function with arguments (including default) and multiple return types for a given problem and call (use) the functions in a Python code.
	\item Rewrite a given program using functions to make re-use of code as much as possible.
	\item Identify and fix errors in passing function arguments and return types.
	\item Evaluate the output of a given function(s).
	\item Evaluate the value of different variables within and outside the function (scoping)
	\item Recursions are \textbf{not} part of the exam.
	\item Develop Python code that use, index, manipulate and search (\verb|in| and \verb|not in|) lists.
	\item Iterating through a list using indexing and \verb|in| operations.
	\item Write a simple list comprehensions with a filter for a given list and a condition.
	\item Deduce the value of a variable after trying to modify a list item and a tuple using an index or a key.
	\item Use of dictionary is \textbf{not} part of the exam.
	\item Develop Taylor series approximation for non-polynomial functions for single variable functions. Write a Python code to solve for the Taylor approximation with relative errors.
	\item Develop Newton-Raphson code to find the root of a function. Compute the tolerance error at each iteration.
\end{enumerate}

You won't be required to write lengthy code (more than 30 lines). I will not penalise for obvious typos and syntax errors in your code (for e.g., missing \verb|:| at the end of function definitions), unless that is what is tested.
\end{document}
